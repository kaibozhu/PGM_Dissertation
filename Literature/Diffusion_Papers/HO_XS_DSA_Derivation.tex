\documentclass[11pt]{article}
\usepackage[margin=1in]{geometry}
\usepackage{amsmath}
\usepackage{amsthm}
\usepackage{amsfonts}

\usepackage{graphicx} 

\usepackage{stmaryrd}

\usepackage{multirow}

\usepackage{rotating}

\newcommand{\fig}[1]{Fig.~\ref{#1}}                      % figure
\newcommand{\tbl}[1]{Table~\ref{#1}}                     % table

\newcommand{\benum}{\begin{equation}}
\newcommand{\eenum}{\end{equation}}

\newcommand{\be}{\begin{equation*}}
\newcommand{\ee}{\end{equation*}}

\newcommand{\bea}{\begin{eqnarray*}}
\newcommand{\eea}{\end{eqnarray*}}

\newcommand{\beanum}{\begin{eqnarray}}
\newcommand{\eeanum}{\end{eqnarray}}

\newcommand{\eqt}[1]{Eq. (\ref{#1})}
\newcommand{\eqts}[1]{Eqs. (\ref{#1})}
\newcommand{\omg}{\ensuremath{ \vec{\Omega}}}
\newcommand{\del}{\ensuremath{ \vec{\nabla} }}


\newcommand{\B}[1]{\ensuremath{{B_{#1} }}}

\newcommand{\p}{\ensuremath{ d}}
\newcommand{\M}{\ensuremath{ \mathbf M}}
\newcommand{\Mw}{\ensuremath{\widehat{\mathbf M}}}

\newcommand{\abs}[1]{\ensuremath{\left\lvert #1 \right\rvert}}
\newcommand{\norm}[1]{\ensuremath{\left\lVert #1 \right\rVert}}

\newcommand{\jmp}[1]{\ensuremath{\left\llbracket #1 \right\rrbracket}}
\newcommand{\avg}[1]{\ensuremath{ \left \{ \hspace{-0.055in} \left \{ #1  \right \}\hspace{-0.055in} \right \} }}

\newcommand{\JT}{\ensuremath{\widetilde{J}(s) }}
\newcommand{\PT}{\ensuremath{\widetilde{\phi}(s) }}
% Equation Punctuation
\newcommand{\pec}{\, ,}
\newcommand{\pep}{\, .}


%newcommand{\eqts}[1]{Eq. \reference{1}\emph{}

\begin{document}

%------------------------------------------------
\author{Peter Maginot}
\date{\today}
\title{Arbitrary Order DFEM Diffusion Discretization \\
with Spatially Varying Cross Section\\
Derived via the M4S Method}
\maketitle
%------------------------------------------------
\section{Diffusion Equations}
We begin with the analytic diffusion equation:
\benum
\frac{\p J}{\p x} + \sigma_a \phi = Q(x) \pep
\label{eq:analytic_diff}
\eenum
We will generate a set of DFEM diffusion equations following the first route proposed by Adams and Martin \cite{M4S}.  

\section{Spatial Discretization}

We approximate the true scalar flux $\phi$ in each cell as a $P$ degree polynomial, $\widetilde{\phi}$,
\benum
\label{eq:phi_def}
\phi(x) \approx   \widetilde{\phi}(x) = \sum_{j=1}^{N_P}{\phi_j\B{j}(x) }, ~~x\in[x_{k-1/2},x_{k+1/2}] \pec
\eenum
where $N_P = P+1$, and
\be
\B{j}(x) = \prod_{\substack{i=1 \\ i\neq j}}^{N_P}{ \frac{s_i - s}{s_i - s_j}} \pep
\ee
To determine the $N_P$ unknowns of $\widetilde{\phi}(x)$, $\phi_j$, within each cell, we follow a Galarkin procedure, succesively multiplying \eqt{eq:analytic_diff} by basis functions $\B{i}(x)$, and integrating, generating $N_P$ moment equations:
\benum
\int_{x_{k-1/2}}^{x_{k+1/2}}{\B{i}(x)\left[\frac{\p J}{\p x} + \sigma_a(x) \phi(x) = Q(x)  \right]dx} \pep
\label{eq:raw_basis}
\eenum

First, we transform to transforming to a generic reference element,
\bea
x &=& x_k + \frac{\Delta x_k}{2} s \\
s &\in& [-1,1] \\
\Delta x_k &=& x_{k+1/2} - x_{k-1/2} \\
x_k &=& \frac{x_{k+1/2} + x_{k-1/2}}{2} \pep
\eea
With this transformation, \eqt{eq:raw_basis} becomes:
\benum
\int_{-1}^{1}{\B{i}(s) \left[ \frac{2}{\Delta x_k}\frac{\p J}{\p s} + \sigma_a(s)\phi(s) = Q(s)  \right]\frac{\Delta x_k}{2}ds} \pep
\eenum
Integrating the first term by parts, we have:
\benum
\label{eq:diff}
\B{i}(1)\widehat{J}_{k+1/2} - \B{i}(-1)\widehat{J}_{k-1/2} + \frac{2}{\Delta x_k}\int_{-1}^1{\frac{\p \B{i}}{\p s} D(s) \frac{\p \phi}{\p s} ~ds} + \frac{\Delta x_k}{2} \int_{-1}^1{\sigma_a(s) \B{i}(s) \phi(s)} \pec
\eenum
where we have made the standard diffusion approximation:
\be
J(x) = -D(x)\frac{\p \phi}{\p x}
\ee
within the cell, and $\widehat{J}_{k\pm1/2}$ denotes the vertex currents.  However, with the DFEM spatial discretization, there is no unique value for the current at each cell edges.  One possible solution is to use a $P_1$ approximation of the angular flux at each cell edge:
\be
\psi(x_{k\pm1/2},\mu) \approx \frac{\phi(x_{k\pm1/2})}{2} + \frac{3\mu}{2}J(x_{k\pm1/2}) \pec
\ee
for which the upwinding scheme used to define the angular flux at cell edges in the transport solution process is well defined.
Looking specifically at $k-1/2$ vertex we have:
\benum
\widetilde{\psi}(x_{k-1/2},\mu) = \left \{ \begin{array}{ll}
\frac{\widetilde{\phi}_{k-1,R}}{2} + \frac{3\mu}{2}\widetilde{J}_{k-1,R} & ~\mu>0 \\
\frac{\widetilde{\phi}_{k,L}}{2} + \frac{3\mu}{2}\widetilde{J}_{k,L}& ~\mu<0
\end{array}
\right. 
\pep
\label{eq:upwind_psi}
\eenum
With the definition of \PT given in \eqt{eq:phi_def} and the diffusion approximation, $\widetilde{\phi}_{k-1,R}$ and $\widetilde{J}_{k-1,R}$ become
\begin{subequations}
\label{eq:int_1}
\beanum
\widetilde{\phi}_{k-1,R} &=& \sum_{j=1}^{N_P}{ \B{j}(1) \phi_{k-1,j} } \\
\widetilde{J}_{k-1,R} &=& -D_{k-1}(x_{k-1/2}) \frac{\p \phi}{\p x} = -\frac{2D_{k-1}(1)}{\Delta x_{k-1}}\frac{\p \phi}{\p s} = -\frac{2D_{k-1}(1)}{\Delta x_{k-1}} \sum_{j=1}^{N_P}{ \frac{\p \B{j}}{\p s} \bigg \lvert_{s=1} \phi_{k-1,j} } \pec 
\eeanum
\end{subequations}
with $\widetilde{\phi}_{k,L}$ and $\widetilde{J}_{k,L}$ being similarly defined as:
\begin{subequations}
\label{eq:int_2}
\beanum
\widetilde{\phi}_{k,L} &=& \sum_{j=1}^{N_P}{ \B{j}(-1) \phi_{k,j} } \\
\widetilde{J}_{k,L} &=& -\frac{2D(-1)}{\Delta x_k}\sum_{j=1}^{N_P}{ \frac{\p \B{j}}{\p s}  \bigg \lvert_{s=-1} \phi_{k,j} } \pep
\eeanum
\end{subequations}
The $\frac{2}{\Delta x}$ terms appear in the $\widetilde{J}$ definitions of \eqt{eq:int_1} and \eqt{eq:int_2} as a result of the change of variables from physical to reference coordinates.
Using the definitions of \eqt{eq:upwind_psi}, we can now define $\widehat{J}_{k-1/2}$.  We will integrate with the same angular quadrature used in our $S_N$ scheme.  
\begin{multline}
\label{eq:jquad}
\widehat{J}_{k-1/2} = \int_{-1}^1{\mu \psi(x_{k-1/2},\mu)~d\mu} \approx \\
\sum_{\substack{d=1 \\ \mu_d > 0}}^{N_{dir}}{w_d\mu_d\left[\frac{\widetilde{\phi}_{k-1,R}}{2} + \frac{3\mu_d}{2}\widetilde{J}_{k-1,R}  \right] } 
+ \sum_{\substack{d=1 \\ \mu_d < 0}}^{N_{dir}}{w_d\mu_d \left[ \frac{\widetilde{\phi}_{k,L}}{2} + \frac{3\mu_d}{2}\widetilde{J}_{k,L} \right] }
\end{multline}
Since we are integrating half range quantities, symmetric quadrature sets defined for $\mu\in[-1,1]$ will not exactly integrate functions over the intervals $\mu \in[-1,0]$ and $\mu\in[0,1]$.
Thus, we introduce $\alpha$:
\benum
\alpha = \sum_{\substack{d=1\\ \mu_d > 0}}^{N_{dir}}{w_d \mu_d} \approx \frac{1}{2} \pep
\eenum
In general, symmetric quadrature sets will integrate even functions of $\mu$ exactly over the half range, so we do not need to introduce a quadrature approximation for this.  We further assume that $\sum_{d=1}^{N_{dir}}{w_d} = 2$.  Performing the quadrature integration of \eqt{eq:jquad} we have
%
%
\benum
\widehat{J}_{k-1/2} = \alpha \frac{\widetilde{\phi}_{k-1,R}}{2} + \frac{\widetilde{J}_{k-1,R}}{2} - \alpha \frac{\widetilde{\phi}_{k,L}}{2} + \frac{\widetilde{J}_{k,L}}{2}
\eenum
and using \eqt{eq:int_1} and \eqt{eq:int_2}, we have:
\begin{multline}
\widehat{J}_{k-1/2} = 
\frac{\alpha}{2}\left[\sum_{j=1}^{N_P}{ \B{j}(1) \phi_{k-1,j} }\right] + 
\frac{1}{2}\left[ -\frac{2D_{k-1}(1)}{\Delta x_{k-1}} \sum_{j=1}^{N_P}{ \frac{\p \B{j}}{\p s} \bigg \lvert_{s=1} \phi_{k-1,j} } \right] \\
- \frac{\alpha}{2}\left[ \sum_{j=1}^{N_P}{ \B{j}(-1) \phi_{k,j} } \right] 
+ \frac{1}{2}\left[-\frac{2D_k(-1)}{\Delta x_k}\sum_{j=1}^{N_P}{ \frac{\p \B{j}}{\p s}  \bigg \lvert_{s=-1} \phi_{k,j} }\right] 
\pep
\end{multline}
%
%
% Stopped here for the night
%
%
When simplified (slightly), this becomes:
\benum
\widehat{J}_{k-1/2} = \frac{1}{2}\sum_{j=1}^{N_P}{\left[\alpha \B{j}(1) -\frac{2}{\Delta x_{k-1}} D_{k-1} \frac{\p \B{j}}{\p s} \bigg \lvert_{s=1}  \right]\phi_{k-1,j} } 
-\frac{1}{2}\sum_{j=1}^{N_P}{\left[ \alpha \B{j}(-1) +\frac{2}{\Delta x_k}D_k(-1)\frac{\p \B{j}}{\p s} \bigg \lvert_{s=-1}  \right]\phi_{k,j}}
\pep
\eenum
Analogously, the equation for $\widehat{J}_{k+1/2}$ is:
\benum
\widehat{J}_{k+1/2} = \frac{1}{2}\sum_{j=1}^{N_P}{\left[\alpha \B{j}(1) - \frac{2}{\Delta x_k}D_{k} \frac{\p \B{j}}{\p s} \bigg \lvert_{s=1}  \right]\phi_{k,j} } 
-\frac{1}{2}\sum_{j=1}^{N_P}{\left[ \alpha \B{j}(-1) + \frac{2}{\Delta x_{k+1}}D_{k+1}(-1)\frac{\p \B{j}}{\p s} \bigg \lvert_{s=-1}  \right]\phi_{k+1,j}}
 \pep
\eenum
%
%
If we consider the $N_P$ moments equations in a given mesh cell at once, we have the following $N_P \times N_P$ system of equations:
\benum
\left[ \mathbf{S}_+\left(\mathbf{J}_{L,k+1} \vec{\phi}_{k+1} + \mathbf{J}_{R,k} \vec{\phi}_{k} \right) -
\mathbf{S}_- \left(\mathbf{J}_{L,k} \vec{\phi}_k + \mathbf{J}_{R,k-1} \vec{\phi}_{k-1} \right) \right]
+ \mathbf{L}\vec{\phi}_k + \Mw_{\sigma_a} \vec{\phi}_k = \M \vec{Q} \pec
\eenum
where we make the following definitions:
\benum
\mathbf{J}_{L,k,1 \dots N_P,j} = -\frac{1}{2}\left[ \frac{2}{\Delta x_k}D_{k}(-1) \frac{\p \B{j}}{\p s}\bigg \lvert_{s=-1} + \alpha \B{j}(-1)  \right]
\eenum
\benum
\mathbf{J}_{R,k,1 \dots N_P,j} = \frac{1}{2}\left[ \alpha \B{j}(1) -\frac{2}{\Delta x_k} D_{k}(1) \frac{\p \B{j}}{\p s}\bigg \lvert_{s=1}  \right]
\eenum
%
\benum
\mathbf{S}_{\pm,ij} = \left \{ \begin{array}{ll} \B{i}(\pm 1) & i=j \\ 0 & \text{otherwise} \end{array} \right. 
\eenum
%
\benum
\mathbf{L}_{ij} = \frac{2}{\Delta x_k}\int_{-1}^{1}{D_k(s) \frac{\p \B{i} }{\p s} \frac{\p B{j}}{\p s} ~ds}
\eenum
\benum
\vec{\phi}_k = \left[ \begin{array}{c} \phi_{1,k} \\ \vdots \\ \phi_{N_P,k} \end{array} \right]\pec
\eenum
\benum
\Mw_{\sigma_a,ij} = \frac{\Delta x}{2} \int_{-1}^1{\sigma_a(s) \B{i}(s) \B{j}(s)~ds} \pec
\eenum
\benum
\M_{ij} = \frac{\Delta x}{2} \int_{-1}^1{\B{i}(s) \B{j}(s)~ds} \pec
\eenum
\benum
\vec{Q} = \left[ \begin{array}{c} Q_{1,k} \\ \vdots \\ Q_{N_P,k} \end{array} \right]\pep
\eenum
In practice, we will approximate the $\mathbf{L}$, \Mw, and \M~matrices using numerical quadrature:
\bea
\M_{ij} &\approx & \frac{\Delta x_k}{2} \sum_{q=1}^{N_q}{w_q \B{i}(s_q) \B{j}(s_q)} \\
\Mw_{\sigma_a,ij} &\approx & \frac{\Delta x_k}{2} \sum_{q=1}^{N_q}{w_q \sigma_a(s_q) \B{i}(s_q) \B{j}(s_q)} \\
\mathbf{L}_{ij} &\approx & \frac{1}{\Delta x_k} \sum_{q=1}^{N_q}{w_q D_k(s_q) \frac{\p \B{i}}{\p s} \bigg \lvert_{s_q}  \frac{\p \B{j}}{\p s} \bigg \lvert_{s_q} }
\eea
If we use numerical quadrature restricted to the DFEM interpolation points, \M ~and $\Mw_{\sigma_a}$ ~become diagonal matrices since,
\benum
\B{i}(s_q) = \left \{ \begin{array}{ll} 1 & s_i  = s_q \\ 0 & \text{otherwise} \end{array} \right. \pep
\eenum
Using self-lumping quadrature, \M~and $\Mw_{\sigma_a}$ are:
\beanum
\M_{ij} &=& \left \{ \begin{array}{ll} w_i & i  = j \\ 0 & \text{otherwise} \end{array} \right. \\
\Mw_{ij,\sigma_a} &=& \left \{ \begin{array}{ll} w_i \sigma_{a}(s_i) & i  = j \\ 0 & \text{otherwise} \end{array} \right.  \pep
\eeanum

\section{Boundary Conditions}

We'll now consider the boundary conditions for our DSA equations.

\subsection{Vacuum (Incident Flux Transport BC)}
For a fixed incident flux transport boundary condition, we do not wish to have any correction to the inward directed flux.  Thus, on the left boundary, $\widehat{J}_{1/2}$ is:
\beanum
\widehat{J}_{1/2} &=& \int_{-1}^1{\mu \psi(x_{1/2},\mu)~d\mu} \approx 0 + \sum_{\substack{d=1 \\\mu_d < 0}}^{N_{dir}}{w_d \mu_d\left[\frac{\widetilde{\phi}_1}{2} + \frac{3\mu_d \widetilde{J}_1}{2}  \right] }\\
\widehat{J}_{1/2} &=& -\frac{1}{2}\left[
\sum_{j=1}^{N_P}{\alpha \B{j}(-1) \phi_{1,j}  +
\frac{2D_1(-1)}{\Delta x}  \sum_{j=1}^{N_P}
\frac{\p \B{j}}{\p s} \bigg \lvert_{s=-1} \phi_{1,j}} \pep
\right]
\eeanum
This make the $N_P$ moment equation in the leftmost cell:
\benum
\left[ \mathbf{S}_+\left(\mathbf{J}_{L,2} \vec{\phi}_{2} + \mathbf{J}_{R,1} \vec{\phi}_{1} \right) -
\mathbf{S}_- \mathbf{J}_{L,1} \vec{\phi}_1  \right]
+ \mathbf{L}\vec{\phi}_k + \Mw_{\sigma_a} \vec{\phi}_k = \M \vec{Q} \pep
\eenum
Similarly on the rightmost cell, the moment equations become:
\benum
\left[ \mathbf{S}_+ \mathbf{J}_{R,N_{cell}} \vec{\phi}_{N_{cell}} -
\mathbf{S}_- \left(\mathbf{J}_{L,N_{cell}} \vec{\phi}_{N_{cell}} + \mathbf{J}_{R,N_{cell}-1} \vec{\phi}_{N_{cell}-1} \right) \right]
+ \mathbf{L}\vec{\phi}_{N_{cell}} + \Mw_{\sigma_a} \vec{\phi}_{N_{cell}} = \M \vec{Q} \pep
\eenum

\subsection{Reflecting (Reflecting Transport BC)}

For reflective transport boundary conditions, we need a reflective DSA boundary condition.  This is implemented most clearly by setting $\widehat{J}_{1/2}=0$, since everything that goes out of the slab is reflected back in, result in a net current of $0$.  The moment equation at the left most and right most cell are then:
\benum
\mathbf{S}_+\left(\mathbf{J}_{L,2} \vec{\phi}_{2} + \mathbf{J}_{R,1} \vec{\phi}_{1} \right) 
+ \mathbf{L}\vec{\phi}_1 + \Mw_{\sigma_a} \vec{\phi}_1 = \M \vec{Q} \pec
\eenum
\benum
\mathbf{S}_- \left(\mathbf{J}_{L,N_{cell}} \vec{\phi}_{N_{cell}} + \mathbf{J}_{R,N_{cell}-1} \vec{\phi}_{N_{cell}-1} \right) 
+ \mathbf{L}\vec{\phi}_{N_{cell}} + \Mw_{\sigma_a} \vec{\phi}_{N_{cell}} = \M \vec{Q} \pec
\eenum

\section{Alternative Use of Integration By Parts}

\benum
\B{i} \left[ \widehat{J}_{out} - \widehat{J}_{in} \right]
\eenum

\benum
\left( \B{i}(1)J_{+,k,k+1/2} + \B{i}(-1) J_{-,k,k-1/2} \right) - \left(\B{i}(-1)J_{+,k-1,k-1/2} + \B{i}(1) J_{-,k+1,k-1/2} \right) 
\eenum
%
%
%
\begin{multline}
\left(\B{i}(1)
 \sum_{\substack{d=1 \\ \mu_d > 0}}^{N_{dir}}{
w_d \mu_d \sum_{j=1}^{N_P}{\phi_{k,j}\left[ \frac{\B{j}(1)}{2} - \frac{3\mu_d}{2} D_k(1)\frac{2}{\Delta x_k} \frac{\p \B{j}}{\p s} \bigg \lvert_{s=1}     \right] }
} \right. \\
\left.
+ \B{i}(-1)
 \sum_{\substack{d=1 \\ \mu_d < 0}}^{N_{dir}}{
w_d \mu_d \sum_{j=1}^{N_P}{\phi_{k,j}\left[ \frac{\B{j}(-1)}{2} - \frac{3\mu_d}{2} D_k(-1)\frac{2}{\Delta x_k} \frac{\p \B{i}}{\p s} \bigg \lvert_{s=-1}      \right] }
}
 \right) \\ 
-
\left(   
\B{i}(-1)\sum_{\substack{d=1 \\ \mu_d > 0}}^{N_{dir}}{
w_d\mu_d \sum_{j=1}^{N_P}{
 \phi_{k-1,j}\left[  \frac{\B{j}(1)}{2} - \frac{3\mu_d}{2} \frac{\Delta x_{k-1}}{2}D_{k-1} \frac{\p B_j}{\p s} \bigg \lvert_{s=1} \right]
}
}
\right.
\\
\left.
+ \B{i}(1) \sum_{\substack{d=1 \\ \mu_d < 0}}^{N_{dir}}{w_d \mu_d \sum_{j=1}^{N_P}{
\phi_{k+1,j}\left[
\frac{\B{j}(-1)}{2} - \frac{2}{\Delta x_{k+1}}D_{k+1}(-1) \frac{\p \B{j}}{\p s}\bigg \lvert_{s=-1}
\right]
}
}
\right)
\end{multline}

\benum
\left( \mathbf{S}_+ \mathbf{J}_{R,k} \vec{\phi}_k + \mathbf{S}_- \mathbf{J}_{L,k} \vec{\phi}_k \right)
-
\left( \mathbf{S}_+ \mathbf{J}_{L,k+1} \vec{\phi}_{k+1} + \mathbf{S}_- \mathbf{J}_{R,k-1} \right) \vec{\phi}_{k-1}
\eenum



\section{Stencil Size}
The stencil of this DSA scheme will be dependent on the DFEM interpolation points selected.  
If there is not a DFEM interpolation point located at each cell vertex, the stencil increases significantly.  
This is caused by $\mathbf{S}_{\pm}$.  
If there is a DFEM interpolation point on each cell edge, then $\mathbf{S}_+ \mathbf{J}_{L,k+1} \vec{\phi}_{k+1}$ will result in a non-zero coefficient of only one $\phi_{k+1,j}$, $\phi_{k+1,1}$ in the $N_P$ moment equation of $\vec{\phi}_k$.  
Assuming there is a DFEM interpolation point at each vertex, in matrix form, the $N_P$ moment equations for $\vec{\phi}_k$ have $N_P \times N_P + 2$ non-zero entries.
However, if there is no DFEM interpolation point at the cell edges, rather than coupling to 1 unknown of each neighboring cell, the moment equations of $\vec{\phi}$ are fully coupled to the neighboring cells, creating a $3(N_P \times N_P)$ diffusion equation stencil.


\section{MIP DSA}

Start with Yaqi and Ragusa's Modified Interior Penalty (MIP) DSA equation:
\begin{subequations}
\label{eq:MIP_formal}
\beanum
b_{MIP}(\phi,\phi^*) = &&\left( \sigma_a\phi,\phi^* \right)_{\cal D} + \left( D\del\phi,\del\phi \right)_{\cal D} \\
&& +\left( \kappa_e \jmp{\phi},\jmp{\phi^*}\right)_{E_h^i} \\
&& +\left(  \jmp{\phi}, \avg{ D\partial_n \phi^*}\right)_{E_h^i} \\
&& +\left( \avg{D\partial_n\phi}, \jmp{\phi^*} \right)_{E_h^i} \\
&& +\left( \kappa_e \phi, \phi^*\right)_{\partial {\cal D}^d} - \frac{1}{2}\left( \phi, D\partial_n \phi^* \right)_{\partial {\cal D}^d} \\
&& -\frac{1}{2}\left( D \partial_n \phi , \phi^* \right)_{\partial {\cal D}^d}
\pep
\eeanum
\end{subequations}
\eqt{eq:MIP_formal} involves a series of integrals/evaluations over all cells and edges in the spatial domain.
We will first consider the interior edges and integrals.  
Since all of the test functions, $\phi^*$ are only non-zero within a given spatial cell $c$, we define the terms of \eqt{eq:MIP_formal} by looking at the integrals in a single cell, $c$, with left and right edges $c-1/2$ and $c+1/2$, respectively.
The volumetric integrals of Eq. (\ref{eq:MIP_formal}a) are the simplest to calculate:
\benum
\left( \sigma_a\phi,\phi^* \right) = \frac{\Delta x_c}{2} \int_{-1}^1{\sigma_a(s)\widetilde{\phi}(s) \B{i}(s)~ds}
\label{eq:vol1}
\eenum
and
\benum
\left( D \del \phi, \del \phi^* \right) = \frac{2}{\Delta x_c} \int_{-1}^1{ D(s) \frac{ \p \widetilde{\phi}}{\p s} \frac{\p \B{i}}{\p s}~ds}\pep
\label{eq:vol2}
\eenum
While Eq. (\ref{eq:MIP_formal}c) - Eq. (\ref{eq:MIP_formal}d) are defined as edge integrations, because \B{i}(s), which are the $\phi^*$ test functions, are only non-zero within a single cell, then \avg{\phi^*} and \jmp{\phi^*} are only non zero on edges $c-1/2$ and $c+1/2$ when \B{i} is defined is non-zero in cell $c$.
Considering the left edge, $c-1/2$ first, $\left( \kappa_e \jmp{\phi},\jmp{\phi^*}\right)$ is:
\benum
\left( \kappa_{c-1/2} \jmp{\phi},\jmp{\phi^*}\right)_{c-1/2} = \kappa_{c-1/2}\left( \widetilde{\phi}_{c}\bigg \lvert_{s=-1} - \widetilde{\phi}_{c-1} \bigg \lvert_{s=1} \right) \left( \B{i}(-1) - 0 \right) \pep
\label{eq:kap_term}
\eenum
%
%
%
The next term, $\left(  \jmp{\phi}, \avg{ D\partial_n \phi^*}\right)$:
\benum
\left(  \jmp{\phi}, \avg{ D\partial_n \phi^*}\right)_{c-1/2} = \left( \widetilde{\phi}_c\bigg \lvert_{s=-1} - \widetilde{\phi}_{c-1} \bigg \lvert_{s=1} \right)\frac{1}{2} \left( 0 + \frac{2D_c }{\Delta x_c} \frac{\p \B{i}}{\p s} \bigg \lvert_{s=-1} \right) \pep
\eenum
%
%
%
The Eq. (\ref{eq:MIP_formal}d) term is:
\benum
\left( \avg{D\partial_n \phi}, \jmp{\phi^*} \right)_{c-1/2} = \frac{1}{2}\left(\frac{2D_{c-1} }{\Delta x_{c-1}} \frac{\p \widetilde{\phi}_{c-1}}{\p s} \bigg \lvert_{s=1} +  \frac{2D_{c}}{\Delta x_{c}} \frac{\p \widetilde{\phi}_{c}}{\p s} \bigg \lvert_{s=-1}  \right) \left( \B{i}(-1) - 0  \right)
\pep
\label{eq:edge3}
\eenum
%
%
%
In \eqt{eq:kap_term}, $\kappa_{c-1/2}$ is defined as:
\benum
\kappa_{c-1/2} = \max\left(\frac{1}{4}, \kappa^{IP}_{c-1/2}  \right) \pec
\eenum
and $\kappa^{IP}_{c-1/2}$ is defined as:
\benum
\kappa_{c-1/2}^{IP} = p_c (p_c+1)\frac{D_c}{\Delta x_c} \bigg \lvert_{s=-1} + p_{c-1} (p_{c-1} + 1)\frac{D_{c-1}}{\Delta x_{c-1}} \bigg \lvert_{s=1} \pec
\eenum
where $p_c$ is the DFEM trial space degree in cell $c$.
We now define the right edge terms quickly:
\beanum
\left( \kappa_{c+1/2} \jmp{\phi},\jmp{\phi^*}\right)_{c+1/2} &=& \kappa_{c+1/2}\left( \widetilde{\phi}_{c+1}\bigg \lvert_{s=-1} - \widetilde{\phi}_{c} \bigg \lvert_{s=1} \right) \left( 0-\B{i}(1) \right) \\
%
%
%
\left(  \jmp{\phi}, \avg{ D\partial_n \phi^*}\right)_{c+1/2} &= &\left( \widetilde{\phi}_{c+1}\bigg \lvert_{s=-1} - \widetilde{\phi}_{c} \bigg \lvert_{s=1} \right) \frac{1}{2}\left( 0 +  \frac{2 D_c }{\Delta x_c} \frac{\p \B{i}}{\p s} \bigg \lvert_{s=1} \right)  \\
%
%
%
\left( \avg{D\partial_n \phi}, \jmp{\phi^*} \right)_{c+1/2} &=& \frac{1}{2}\left(\frac{2D_{c} }{\Delta x_{c}} \frac{\p \widetilde{\phi}_{c}}{\p s} \bigg \lvert_{s=1} +  \frac{2D_{c+1} }{\Delta x_{c+1}} \frac{\p \widetilde{\phi}_{c+1}}{\p s} \bigg \lvert_{s=-1}  \right) \left( 0- \B{i}(1) \right)
\eeanum

In the above, we have implicitly assumed that the edge normal vector, $\vec{n}$ is oriented in the $+x$ direction.
On the cell interior, this orientation is arbitrary, so long as it is consistent.  
However, on the global boundaries, the edge normal vector must be oriented outward. 
On the left boundary, this is in the $-x$ direction, and on the right boundary, in $+x$ direction.

It is possible to consider the MIP equations in matrix form within each cell.  First, let us define the following $N_P \times N_P$ diagonal matrices:
\beanum
\mathbf{B}_{L,ii} &=& \B{i}(-1) \\
\mathbf{B}_{R,ii} &=& \B{i}(1)  \\
\widehat{\mathbf B}_{L,ii} &=& \frac{\p \B{i}}{\p s} \bigg \lvert_{s=-1}\\
\widehat{\mathbf B}_{R,ii} &=& \frac{\p \B{i}}{\p s} \bigg \lvert_{s= 1} \pep
\eeanum
We also define the following $N_P \times N_P$ matrices:
\benum
\begin{array}{c}
\mathbf{P}_{L,i} \vspace{0.15in}\\
\mathbf{P}_{R,i} \vspace{0.2in}\\
\widehat{\mathbf P}_{L,i} \vspace{0.25in}\\
\widehat{\mathbf P}_{R,i}
\end{array}
\begin{array}{c}
= \vspace{0.15in} \\
= \vspace{0.20in} \\
= \vspace{0.25in} \\
= \\
\end{array}
\begin{array}{cccccc}
\Big [ & B_1(-1) & B_2(-1) & \dots & B_{N_P}(-1) & \Big ] \vspace{0.1in}\\
\Big [ & B_1(1) & B_2(1) & \dots & B_{N_P}(1) & \Big ] \vspace{0.1in}\\
\bigg [ & \frac{\p B_1}{\p s}\bigg \lvert_{s=-1} & \frac{\p B_2}{\p s}\bigg \lvert_{s=-1} & \dots & \frac{\p B_{N_P}}{\p s}\bigg \lvert_{s=-1} & \bigg ] \vspace{0.1in}\\
\bigg [ & \frac{\p B_1}{\p s}\bigg \lvert_{s=1} & \frac{\p B_2}{\p s}\bigg \lvert_{s=1} & \dots & \frac{\p B_{N_P}}{\p s}\bigg \lvert_{s=1} & \bigg ]
\end{array} \pep
\eenum
We note that $\mathbf{P}_L$ and $\mathbf{P}_R$ have only $N_P$ non-zero entries if there are interpolation points located at $s=\pm1$.
The volumetric MIP moment equations, \eqt{eq:vol1} and \eqt{eq:vol2} become:
\benum
\label{eq:vol}
\frac{\Delta x_c}{2} \mathbf{R}_{\sigma_a} \vec{\phi}_c + \frac{2}{\Delta x_c}\mathbf{L}_{D} \vec{\phi}_c \pec 
\eenum
where:
\beanum
\mathbf{R}_{\sigma_a,ij} &=& \int_{-1}^1{\sigma_a(s) \B{i}(s) \B{j}(s) ~ds} \\
\mathbf{L}_{D,ij} &=& \int_{-1}^1{ D(s) \frac{\p \B{i}}{\p s}  \frac{\p \B{j}}{\p s}~ds} \pep
\eeanum
The left edge terms in matrix form are:
\beanum
\left( \kappa_{c-1/2} \jmp{\phi},\jmp{\phi^*}\right)_{c-1/2} &=& \kappa_{c-1/2}\mathbf{B}_L \mathbf{P}_L \vec{\phi}_{c} - \kappa_{c-1/2}\mathbf{B}_L \mathbf{P}_R \vec{\phi}_{c-1} \\
\left(  \jmp{\phi}, \avg{ D\partial_n \phi^*}\right)_{c-1/2} &=& \frac{D_c}{\Delta x_c} \bigg \lvert_{s=-1} \widehat{\mathbf B}_L \left[ \mathbf{ P}_L \vec{\phi}_{c} - \mathbf{P}_R \vec{\phi}_{c-1}  \right] \\
\left( \avg{D\partial_n \phi}, \jmp{\phi^*} \right)_{c-1/2} &=& \mathbf{B}_L\left( \frac{D_c}{\Delta x_c}\bigg \lvert_{s=-1} \widehat{\mathbf P}_L\vec{\phi}_c + \frac{D_{c-1}}{\Delta x_{c-1}} \bigg \lvert_{s=1} \widehat{\mathbf P}_R \vec{\phi}_{c-1}   \right)\pep
\eeanum
The right edge terms are:
\beanum
\left( \kappa_{c+1/2} \jmp{\phi},\jmp{\phi^*}\right)_{c+1/2} &=& -\kappa_{c+1/2}\widehat{\mathbf B}_R\left[ \mathbf{P}_L \vec{\phi}_{c+1} - \mathbf{P}_R \vec{\phi}_c \right] \\
\left(  \jmp{\phi}, \avg{ D\partial_n \phi^*}\right)_{c+1/2} &=& \frac{D_c}{\Delta x_c }\bigg \lvert_{s=1} \widehat{\mathbf B}_R\left[ \mathbf{P}_L \vec{\phi}_{c+1} - \mathbf{P}_R \vec{\phi}_c  \right] \\
\left( \avg{D\partial_n \phi}, \jmp{\phi^*} \right)_{c+1/2} &=& -\mathbf{ B}_R\left[ \frac{D_c}{\Delta x_c} \bigg \lvert_{s=1} \widehat{\mathbf P}_R \vec{\phi}_c+ \frac{D_{c+1}}{\Delta x_{c+1}} \bigg \lvert_{s=-1} \widehat{\mathbf P}_L \vec{\phi}_{c+1} \right]\pep
\eeanum
%
%\begin{multline}
%\kappa_{c-1/2} \mathbf{B}_L \left(\mathbf{P}_L \vec{\phi}_c - \mathbf{P}_R \vec{\phi}_{c-1}  \right)
%+   \frac{D_c}{\Delta x_c} \bigg \lvert_{s=-1} \widehat{\mathbf B}_L \left( \mathbf{P}_L\vec{\phi}_c - \mathbf{P}_R \vec{\phi}_{c-1} \right) \\
%+ \mathbf{B}_L \left( \frac{D_{c-1}}{\Delta x_{c-1}} \bigg \lvert_{s=1} \widehat{\mathbf P}_R \vec{\phi}_{c-1} + \frac{D_c}{\Delta x_c} \bigg \lvert_{s=-1}  \widehat{\mathbf P}_L \vec{\phi}_c  \right) \pep
%\label{eq:left_e}
%\end{multline}
%
The interior cell MIP equations are:
\begin{multline*}
\left( \sigma_a\phi,\phi^* \right)_c + \left( D\del\phi,\del\phi \right)_c
 +\left( \kappa_{c-1/2} \jmp{\phi},\jmp{\phi^*}\right)_{c-1/2} 
 +\left(  \jmp{\phi}, \avg{ D\partial_n \phi^*}\right)_{c-1/2} 
 +\left( \avg{D\partial_n\phi}, \jmp{\phi^*} \right)_{c-1/2} \\
  +\left( \kappa_{c+1/2} \jmp{\phi},\jmp{\phi^*}\right)_{c+1/2} 
 +\left(  \jmp{\phi}, \avg{ D\partial_n \phi^*}\right)_{c+1/2} 
 +\left( \avg{D\partial_n\phi}, \jmp{\phi^*} \right)_{c+1/2} \pep
\end{multline*}
%
%
Substituting in our definitions
\begin{multline}
\left[ -\kappa_{c-1/2}\mathbf{B}_L\mathbf{P}_R - \frac{D_c}{\Delta x_c} \bigg \lvert_{s=-1} \widehat{\mathbf B}_L  \mathbf{P}_R + \mathbf{B}_L \frac{D_{c-1}}{\Delta x_{c-1}}\bigg \lvert_{s=1} \widehat{\mathbf P}_R \right]\vec{\phi}_{c-1} \\
%
%
\left[\frac{\Delta x_c}{2}\mathbf{R}_{\sigma_a} + \frac{2}{\Delta x_c}\mathbf{L}_D + \kappa_{c-1/2}\mathbf{B}_L\mathbf{P}_L + \frac{D_c}{\Delta x_c} \bigg \lvert_{s=-1} \widehat{\mathbf B}_L \mathbf{P}_L  + \mathbf{B}_L \frac{D_c}{\Delta x_c} \bigg \lvert_{s=-1} \widehat{\mathbf P}_L \dots \right. \\
\left. \kappa_{c+1/2}\mathbf{B}_R \mathbf{P}_R - \frac{D_c}{\Delta x_c}\bigg \lvert_{s=1}\widehat{\mathbf B}_R \mathbf{P}_R -\mathbf{ B}_R \frac{D_c}{\Delta x_c}\bigg \lvert_{s=1} \widehat{\mathbf P}_R \right] \vec{\phi}_c \\
%
%
\left[ -\kappa_{c+1/2}\mathbf{B}_R \mathbf{P}_L + \frac{D_c}{\Delta x_c}\bigg \lvert_{s=1}\widehat{\mathbf B}_R \mathbf{P}_L - \mathbf{ B}_R \frac{D_{c+1}}{\Delta x_{c+1}} \bigg \lvert_{s=-1} \widehat{\mathbf P}_L \right] \vec{\phi}_{c+1}
\end{multline}
%
%%%%The right edge terms are:
%%%%\begin{multline}
%%%%-\kappa_{c+1/2}\mathbf{B}_R\left( \mathbf{P}_L \vec{\phi}_{c+1} - \mathbf{P}_R \vec{\phi}_{c} \right) +
%%%%\frac{D_c}{\Delta x_c} \bigg\lvert_{s=1} \widehat{\mathbf B}_R \left(\mathbf{P}_L \vec{\phi}_{c+1} - \mathbf{P}_R \vec{\phi}_c  \right) \\
%%%%-\mathbf{B}_R \left(\frac{D_c}{ \Delta x_c} \bigg \lvert_{s=1} \widehat{\mathbf P}_R \vec{\phi}_c + \frac{D_{c+1} }{\Delta x_{c+1}} \bigg \lvert_{s=-1} \widehat{\mathbf P}_L \vec{\phi}_{c+1}  \right) \pep
%%%%\label{eq:right_e}
%%%%\end{multline}
%%%%%
%%%%%
%%%%%
%%%%Combining \eqt{eq:vol}, \eqt{eq:left_e}, and \eqt{eq:right_e} and simplifying we have:
%%%%%
%%%%%
%%%%\begin{multline}
%%%%-\left[ \kappa_{c-1/2} \mathbf{B}_L \mathbf{P}_R + \frac{ D_c}{\Delta x_c}\big \lvert_{s=-1}\widehat{\mathbf B}_L \mathbf{P}_R
%%%%-\frac{D_{c-1}}{\Delta x_{c-1}}\bigg \lvert_{s=1} \mathbf{B}_L \widehat{\mathbf P}_R \right]\vec{\phi}_{c-1} \\
%%%%%
%%%%%
%%%%+\left[\frac{\Delta x_c}{2} \mathbf{R}_{\sigma_a} + \frac{2}{\Delta x_c}\mathbf{L}_D + \kappa_{c-1/2}\mathbf{B}_L \mathbf{P}_L 
%%%%+ \frac{D_c}{\Delta x_c}\bigg \lvert_{s=-1}\left( \widehat{\mathbf B}_L \mathbf{P}_L + \mathbf{B}_L \widehat{\mathbf P}_L \right)
%%%% \dots \right. \\ \left. 
%%%% + \kappa_{c+1/2} \mathbf{B}_R \mathbf{P}_R - \frac{D_c}{\Delta x_c}\bigg \lvert_{s=1}\left(  \widehat{\mathbf B}_R\mathbf{P}_R + \mathbf{B}_R \widehat{\mathbf P}_R \right)\right]\vec{\phi}_c \\
%%%%%
%%%%%
%%%%-\left[\kappa_{c+1/2}\mathbf{B}_R \mathbf{P}_L - \frac{D_c}{\Delta x_c} \bigg\lvert_{s=1}\widehat{\mathbf B}_R \mathbf{P}_L +
%%%%\frac{D_{c+1}}{\Delta x_{c+1}} \bigg \lvert_{s=-1}\mathbf{B}_R \widehat{\mathbf P}_L \right]\vec{\phi}_{c+1} \pep
%%%%\end{multline}

The boundary cells have only one interior edge, but the boundary terms come into play.
With this in mind, we handle the domain boundary terms of \eqt{eq:MIP_formal}.
Starting with the left boundary:
\beanum
\left( \kappa_{1/2} \phi, \phi^*\right)_{1/2} &=& \kappa_{1/2}\left(\widetilde{\phi}_1(-1) \right) \B{i}(-1) \\
\frac{1}{2}\left( \phi, D\partial_n \phi^* \right)_{1/2} &=&  \frac{1}{2}\widetilde{\phi}_1(-1) \left(-\frac{2 D_1}{\Delta x_1} \frac{\p \B{i}}{\p s} \bigg \lvert_{s=-1} \right)\\
\frac{1}{2}\left( D \partial_n \phi , \phi^* \right)_{1/2} &=& \frac{1}{2}\left(- \frac{2D_1}{\Delta x_1} \frac{\p \widetilde{\phi}_1}{\p s} \bigg \lvert_{s=-1} \right) \B{i}(-1) \\
\kappa_{1/2} &=& \max \left( \frac{1}{4},\kappa_{1/2}^{IP} \right) \\
\kappa_{1/2}^{IP} &=& 2 p_1 (p_1 + 1)\frac{D_1}{\Delta x_1}\bigg \lvert_{s=-1} \pep
\eeanum
The leftmost cell moment equations are:
\begin{multline*}
\left( \sigma_a\phi,\phi^* \right)_1 + \left( D\del\phi,\del\phi \right)_1
  +\left( \kappa_{3/2} \jmp{\phi},\jmp{\phi^*}\right)_{3/2} 
 +\left(  \jmp{\phi}, \avg{ D\partial_n \phi^*}\right)_{3/2} \\
 +\left( \avg{D\partial_n\phi}, \jmp{\phi^*} \right)_{3/2} 
 + (\kappa_{1/2} \phi, \phi^*) - \frac{1}{2}\left( \phi,D\partial_n \phi^*  \right) - \frac{1}{2}\left( D\partial_n \phi, \phi^* \right) \pep
\end{multline*}
%
%
Written in matrix form:
\begin{multline}
\left[ \frac{\Delta x_1}{2}\mathbf{R}_{\sigma_a} + \frac{2}{\Delta x_1}\mathbf{L} + \kappa_{3/2} \mathbf{ B}_R \mathbf{P}_R - \frac{D_1}{\Delta x_1} \bigg \lvert_{s=1} \widehat{\mathbf B}_R  \mathbf{P}_R
-\mathbf{B}_R \frac{D_1}{\Delta x_1}\bigg \lvert_{s=1} \widehat{\mathbf P}_R \right. \\
 \left. + \kappa_{1/2} \mathbf{B}_L \mathbf{P}_L + \frac{D_1}{\Delta x_1} \bigg \lvert_{s=-1} \widehat{\mathbf B}_L \mathbf{P}_L + \frac{D_1}{\Delta x_1} \bigg \lvert_{s=-1} \mathbf{B}_L \widehat{\mathbf P}_L \right]\vec{\phi}_1 \\
%
+ \left[ -\kappa_{3/2} \mathbf{ B}_R \mathbf{P}_L + \frac{D_1}{\Delta x_1} \bigg \lvert_{s=1} \widehat{\mathbf B}_R \mathbf{P}_L  
-\mathbf{B}_R \frac{D_2}{\Delta x_2} \bigg \lvert_{s=-1} \widehat{\mathbf P}_L \right] \vec{\phi}_2
\end{multline}
%
If instead we use Yaqi's form, not the M\&C form (our what I assumed), the moment equations are:
\begin{multline*}
\left( \sigma_a\phi,\phi^* \right)_1 + \left( D\del\phi,\del\phi \right)_1
  +\left( \kappa_{3/2} \jmp{\phi},\jmp{\phi^*}\right)_{3/2} 
 +\left(  \jmp{\phi}, \avg{ D\partial_n \phi^*}\right)_{3/2} \\
 +\left( \avg{D\partial_n\phi}, \jmp{\phi^*} \right)_{3/2} 
 + (\kappa_{1/2} \phi, \phi^*) - \frac{1}{2}\left( \phi,D\partial_n \phi^*  \right) - \frac{1}{2}\left( D\partial_n \phi, \phi \right) \pep
\end{multline*}
%
We evaluate the different boundary term as:
\benum
(D\partial_n \phi,\phi) = -\frac{2 D_1}{\Delta x_1}\frac{d \widetilde{\phi} }{\p s} \bigg \lvert_{s=-1}  \widetilde{\phi}_1(-1) \pec
\eenum
%
Written in matrix form, the leftmost moment cells are then:
\begin{multline}
\left[ \frac{\Delta x_1}{2}\mathbf{R}_{\sigma_a} + \frac{2}{\Delta x_1}\mathbf{L} + \kappa_{3/2} \mathbf{ B}_R \mathbf{P}_R - \frac{D_1}{\Delta x_1} \bigg \lvert_{s=1} \widehat{\mathbf B}_R  \mathbf{P}_R
-\mathbf{B}_R \frac{D_1}{\Delta x_1}\bigg \lvert_{s=1} \widehat{\mathbf P}_R \right. \\
 \left. + \kappa_{1/2} \mathbf{B}_L \mathbf{P}_L + \frac{D_1}{\Delta x_1} \bigg \lvert_{s=-1} \widehat{\mathbf B}_L \mathbf{P}_L + \mathbf{E}_{left} \right]\vec{\phi}_1 \\
%
+ \left[ -\kappa_{3/2} \mathbf{ B}_R \mathbf{P}_L + \frac{D_1}{\Delta x_1} \bigg \lvert_{s=1} \widehat{\mathbf B}_R \mathbf{P}_L  
-\mathbf{B}_R \frac{D_2}{\Delta x_2} \bigg \lvert_{s=-1} \widehat{\mathbf P}_L \right] \vec{\phi}_2 \pec
\end{multline}
with $N_P \times N_P$ matrix, $\mathbf{E}_{left}$, defined as:
\benum
E_{left,i} = -\frac{D_1}{\Delta x_1} \bigg \lvert_{s=-1} \left[ \frac{d B_1}{d s}\bigg \lvert_{s=-1} B_1(-1) \dots \frac{d B_{N_P}}{d s}\bigg \lvert_{s=-1} B_{N_P}(-1) \right] 
\eenum
%
%
%
%%
Now considering the rightmost cell, the MIP moment equations are:
\begin{multline}
\kappa_{N-1/2} \mathbf{B}_L \left(\mathbf{P}_L \vec{\phi}_N - \mathbf{P}_R \vec{\phi}_{N-1}  \right)
+   \frac{D_N}{\Delta x_N} \bigg \lvert_{s=-1} \widehat{\mathbf B}_L \left( \mathbf{P}_L\vec{\phi}_N - \mathbf{P}_R \vec{\phi}_{N-1} \right) \\
+ \mathbf{B}_L \left( \frac{D_{N-1}}{\Delta x_{N-1}} \bigg \lvert_{s=1} \widehat{\mathbf P}_R \vec{\phi}_{N-1} + \frac{D_N}{\Delta x_N} \bigg \lvert_{s=-1}  \widehat{\mathbf P}_L \vec{\phi}_N  \right) \pec
\end{multline}
where $N$ is the total number of mesh cells.  The right boundary edge terms are:
\beanum
\left( \kappa_{N+1/2} \phi, \phi^*\right)_{N+1/2} &=& \kappa_{N+1/2}\left(\widetilde{\phi}_N(1) \right) \B{i}(1) \\
%
%
\frac{1}{2}\left( \phi, D\partial_n \phi^* \right)_{N+1/2} &=&  \frac{1}{2}\widetilde{\phi}_N(1) \left(\frac{2 D_N}{\Delta x_N} \frac{\p \B{i}}{\p s} \bigg \lvert_{s=1} \right)\\
%
%
\frac{1}{2}\left( D \partial_n \phi , \phi^* \right)_{N+1/2} &=& \frac{1}{2}\left( \frac{2D_N}{\Delta x_N} \frac{\p \widetilde{\phi}_N}{\p s} \bigg \lvert_{s=1} \right) \B{i}(1) \\
%
%
\kappa_{N+1/2} &=& \max \left( \frac{1}{4},\kappa_{N+1/2}^{IP} \right) \\
%
%
\kappa_{N + 1/2}^{IP} &=& 2 p_N \frac{D_N(1)}{\Delta x_N} \pec
\eeanum
Combing all of the edge and cell integral equations together, we have:
\begin{multline}
-\left[\kappa_{N-1/2}\mathbf{B}_L\mathbf{P}_R + \frac{D_N}{\Delta x_N}\bigg \lvert_{s=-1} \widehat{\mathbf B}_L \mathbf{P}_R - \frac{D_{N-1}}{\Delta x_{N-1} }\bigg \lvert_{s=1} \mathbf{B}_L \widehat{\mathbf P }_R \right]\vec{\phi}_{N-1} \\
\left[ \frac{\Delta x_N}{2}\mathbf{R}_{\sigma_a} + \frac{2}{\Delta x_N} \mathbf{L}_D + \kappa_{N-1/2}\mathbf{B}_L\mathbf{P}_L + \frac{D_N}{\Delta x_N} \bigg \lvert_{s=-1} \widehat{\mathbf B}_L\mathbf{P}_L + \frac{D_N}{\Delta x_N}\bigg \lvert_{s=-1} \mathbf{B}_L\widehat{\mathbf P}_L  \dots \right. \\
\left. + \kappa_{N+1/2}\mathbf{B}_R\mathbf{P}_R - \frac{D_N}{\Delta x_N}\bigg \lvert_{s=1}\widehat{\mathbf B}_R \mathbf{P}_R - \frac{D_N}{\Delta x_N}\bigg \lvert_{s=1}\mathbf{B}_R \widehat{\mathbf P}_R   \right]\vec{\phi}_N
\end{multline}

Using Yaqi's last boundary term, the rightmost edge terms are:
\benum
\left( D \partial_n \phi , \phi \right)_{N+1/2} = \left( \frac{2D_N}{\Delta x_N} \frac{\p \widetilde{\phi}_N}{\p s} \bigg \lvert_{s=1} \right) \widetilde{\phi}(1) 
\eenum
The rightmost cell moment equations are then:
\begin{multline*}
\left( \sigma_a\phi,\phi^* \right)_N + \left( D\del\phi,\del\phi \right)_N
 +\left( \kappa_{c-1/2} \jmp{\phi},\jmp{\phi^*}\right)_{N-1/2} 
 +\left(  \jmp{\phi}, \avg{ D\partial_n \phi^*}\right)_{N-1/2} 
 +\left( \avg{D\partial_n\phi}, \jmp{\phi^*} \right)_{N-1/2} \\
  +\left( \kappa_{N+1/2} \phi,\phi^*\right)_{N+1/2}
 -\frac{1}{2}\left( \phi,  D\partial_n \phi^*\right)_{N+1/2} 
 -\frac{1}{2}\left( D\partial_n\phi, \phi \right)_{N+1/2} \pep
\end{multline*}
Writing these in matrix form:
\begin{multline}
\left[ -\kappa_{N-1/2}\mathbf{B}_L\mathbf{P}_R - \frac{D_N}{\Delta x_N}\bigg \lvert_{s=-1} \widehat{\mathbf B}_L \mathbf{P}_R + \mathbf{B}_L \frac{D_{N-1}}{\Delta x_{N-1}} \bigg\lvert_{s=1} \widehat{\mathbf P}_R \right] \vec{\phi}_{N-1} \\
\left[\frac{\Delta x_n}{2} \mathbf{R}_{\sigma_a} + \frac{2}{\Delta x_N}\mathbf{L} +  \kappa_{N-1/2} \mathbf{B}_L \mathbf{P}_L + \frac{D_N}{\Delta x_N} \bigg \lvert_{s=-1} \widehat{\mathbf B}_L \mathbf{P}_L + \mathbf{B}_L \frac{D_N}{\Delta x_N}\bigg \lvert_{s=-1}\widehat{\mathbf P}_L \right.
 \\ \left. +\kappa_{N+1/2}\mathbf{B}_R \mathbf{P}_R - \frac{D_N}{\Delta x_N} \bigg \lvert_{s=1}\widehat{\mathbf B}_R \mathbf{P}_R - \frac{D_N}{\Delta x_N} \mathbf{E}_{right} \right] \vec{\phi}_N \pec
\end{multline}
where we have the $N_P \times N_P$ matrix $\mathbf{E}_{right}$ as:
\benum
\mathbf{E}_{right,i} = \left[ \frac{d B_1}{d s} \bigg \lvert_{s=1} B_1(1) \dots 
\frac{d B_{N_P}}{d s} \bigg \lvert_{s=1} B_{N_P}(1) \right]
\eenum

\begin{thebibliography}{20}

\bibitem{M4S} M. L. Adams and W. R. Martin , ``Diffusion Synthetic Acceleration of Discontinuous Finite Element Transport Iterations,'' {\it Nuclear Science and Engineering}, {\bf 111}, pp. 145-167 (1992).

\end{thebibliography}

\end{document}