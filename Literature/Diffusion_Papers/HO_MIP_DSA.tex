\documentclass[11pt]{article}
\usepackage[margin=1in]{geometry}
\usepackage{amsmath}
\usepackage{amsthm}
\usepackage{amsfonts}

\usepackage{graphicx} 

\usepackage{stmaryrd}

\usepackage{multirow}

\usepackage{rotating}

\newcommand{\fig}[1]{Fig.~\ref{#1}}                      % figure
\newcommand{\tbl}[1]{Table~\ref{#1}}                     % table

\newcommand{\benum}{\begin{equation}}
\newcommand{\eenum}{\end{equation}}

\newcommand{\be}{\begin{equation*}}
\newcommand{\ee}{\end{equation*}}

\newcommand{\bea}{\begin{eqnarray*}}
\newcommand{\eea}{\end{eqnarray*}}

\newcommand{\beanum}{\begin{eqnarray}}
\newcommand{\eeanum}{\end{eqnarray}}

\newcommand{\eqt}[1]{Eq. (\ref{#1})}
\newcommand{\eqts}[1]{Eqs. (\ref{#1})}
\newcommand{\omg}{\ensuremath{ \vec{\Omega}}}
\newcommand{\del}{\ensuremath{ \vec{\nabla} }}


\newcommand{\B}[1]{\ensuremath{{B_{#1} }}}

\newcommand{\p}{\ensuremath{ d}}
\newcommand{\M}{\ensuremath{ \mathbf M}}
\newcommand{\Mw}{\ensuremath{\widehat{\mathbf M}}}

\newcommand{\abs}[1]{\ensuremath{\left\lvert #1 \right\rvert}}
\newcommand{\norm}[1]{\ensuremath{\left\lVert #1 \right\rVert}}

\newcommand{\jmp}[1]{\ensuremath{\left\llbracket #1 \right\rrbracket}}
\newcommand{\avg}[1]{\ensuremath{ \left \{ \hspace{-0.055in} \left \{ #1  \right \}\hspace{-0.055in} \right \} }}

\newcommand{\JT}{\ensuremath{\widetilde{J}(s) }}
\newcommand{\PT}{\ensuremath{\widetilde{\phi}(s) }}
% Equation Punctuation
\newcommand{\pec}{\, ,}
\newcommand{\pep}{\, .}


%newcommand{\eqts}[1]{Eq. \reference{1}\emph{}

\begin{document}

%------------------------------------------------
\author{Peter Maginot}
\date{\today}
\title{Arbitrary Order MIP DFEM Diffusion Discretization \\
with Spatially Varying Cross Section}
\maketitle
%------------------------------------------------
\section{MIP Bilinear Forms}

\subsection{M\&C Form}
The form given by Turcksin and Ragusa at the 2013 M\&C meeting:
\bea
b_{MIP}(\phi,\phi^*) = &&\left( \sigma_a\phi,\phi^* \right)_{\cal D} + \left( D\del\phi,\del\phi \right)_{\cal D} \\
&& +\left( \kappa_e \jmp{\phi},\jmp{\phi^*}\right)_{E_h^i} \\
&& +\left(  \jmp{\phi}, \avg{ D\partial_n \phi^*}\right)_{E_h^i} \\
&& +\left( \avg{D\partial_n\phi}, \jmp{\phi^*} \right)_{E_h^i} \\
&& +\left( \kappa_e \phi, \phi^*\right)_{\partial {\cal D}^d} - \frac{1}{2}\left( \phi, D\partial_n \phi^* \right)_{\partial {\cal D}^d} \\
&& -\frac{1}{2}\left( D \partial_n \phi , \phi^* \right)_{\partial {\cal D}^d}
\pep
\eea

\subsection{NS\&E Form}
The MIP form given in the 2010 NS\&E article by Wang and Ragusa:
\bea
b_{MIP}(\phi,\phi^*) = &&\left( \sigma_a\phi,\phi^* \right)_{\cal D} + \left( D\del\phi,\del\phi \right)_{\cal D} \\
&& +\left( \kappa_e \jmp{\phi},\jmp{\phi^*}\right)_{E_h^i} \\
&& +\left(  \jmp{\phi}, \avg{ D\partial_n \phi^*}\right)_{E_h^i} \\
&& +\left( \avg{D\partial_n\phi}, \jmp{\phi^*} \right)_{E_h^i} \\
&& +\left( \kappa_e \phi, \phi^*\right)_{\partial {\cal D}^d} - \frac{1}{2}\left( \phi, D\partial_n \phi^* \right)_{\partial {\cal D}^d} \\
&& -\frac{1}{2}\left( D \partial_n \phi , \phi \right)_{\partial {\cal D}^d}
\pep
\eea

\section{Shared Components}

\bea
b_{MIP}(\phi,\phi^*) = &&\left( \sigma_a\phi,\phi^* \right)_{\cal D} + \left( D\del\phi,\del\phi \right)_{\cal D} \\
&& +\left( \kappa_e \jmp{\phi},\jmp{\phi^*}\right)_{E_h^i} \\
&& +\left(  \jmp{\phi}, \avg{ D\partial_n \phi^*}\right)_{E_h^i} \\
&& +\left( \avg{D\partial_n\phi}, \jmp{\phi^*} \right)_{E_h^i} \\
&& +\left( \kappa_e \phi, \phi^*\right)_{\partial {\cal D}^d} - \frac{1}{2}\left( \phi, D\partial_n \phi^* \right)_{\partial {\cal D}^d} 
\eea

We will first consider the shared components of both possible MIP derivations.
Since all of the test functions, $\phi^*$, are only non-zero within a given spatial cell $c$, it is convenient to form the bilinear operator cell by cell, we define the equations for a single cell, $c$, with left and right edges $c-1/2$ and $c+1/2$, respectively.
Noting that $\phi^* = \sum{B_i(s)}$, we can write down one equation for each $B_i(s)$.

\subsection{Interior Terms}

The volumetric integrals are
\benum
\left( \sigma_a\phi,\phi^* \right) = \frac{\Delta x_c}{2} \int_{-1}^1{\sigma_a(s)\widetilde{\phi}(s) \B{i}(s)~ds}
\label{eq:vol1}
\eenum
and
\benum
\left( D \del \phi, \del \phi^* \right) = \frac{\Delta x_c}{2} \int_{-1}^1{ D(s) \frac{2}{\Delta x_c} \frac{ \p \widetilde{\phi}}{\p s}~ \frac{2}{\Delta x_c} \frac{\p \B{i}}{\p s}~ds} \pep
\label{eq:vol2}
\eenum

We now look at the non-boundary edge pieces.
Considering the left edge, $c-1/2$, first:
\begin{subequations}
\beanum
\left( \kappa_{c-1/2} \jmp{\phi},\jmp{\phi^*}\right)_{c-1/2} &=& \kappa_{c-1/2}\left( \widetilde{\phi}_{c}\bigg \lvert_{s=-1} - \widetilde{\phi}_{c-1} \bigg \lvert_{s=1} \right) \left( \B{i}(-1) - 0 \right) \pep
\label{eq:kap_term}
%
\\
%
\left(  \jmp{\phi}, \avg{ D\partial_n \phi^*}\right)_{c-1/2} &=& \left( \widetilde{\phi}_c\bigg \lvert_{s=-1} - \widetilde{\phi}_{c-1} \bigg \lvert_{s=1} \right)\frac{1}{2} \left( 0 + \frac{2D_c }{\Delta x_c} \frac{\p \B{i}}{\p s} \bigg \lvert_{s=-1} \right) \pep
%
\\
%
\left( \avg{D\partial_n \phi}, \jmp{\phi^*} \right)_{c-1/2} &=& \frac{1}{2}\left(\frac{2D_{c-1} }{\Delta x_{c-1}} \frac{\p \widetilde{\phi}_{c-1}}{\p s} \bigg \lvert_{s=1} +  \frac{2D_{c}}{\Delta x_{c}} \frac{\p \widetilde{\phi}_{c}}{\p s} \bigg \lvert_{s=-1}  \right) \left( \B{i}(-1) - 0  \right)
\pep
\label{eq:edge3}
\eeanum
\end{subequations}
%
%
%
In \eqt{eq:kap_term}, $\kappa_{c-1/2}$ is defined as:
\benum
\kappa_{c-1/2} = \max\left(\frac{1}{4}, \kappa^{IP}_{c-1/2}  \right) \pec
\eenum
and $\kappa^{IP}_{c-1/2}$ is defined as:
\benum
\kappa_{c-1/2}^{IP} = p_c (p_c+1)\frac{D_c}{\Delta x_c} \bigg \lvert_{s=-1} + p_{c-1} (p_{c-1} + 1)\frac{D_{c-1}}{\Delta x_{c-1}} \bigg \lvert_{s=1} \pec
\eenum
where $p_c$ is the DFEM trial space degree in cell $c$.

We now define the right edge terms quickly:
\beanum
\left( \kappa_{c+1/2} \jmp{\phi},\jmp{\phi^*}\right)_{c+1/2} &=& \kappa_{c+1/2}\left( \widetilde{\phi}_{c+1}\bigg \lvert_{s=-1} - \widetilde{\phi}_{c} \bigg \lvert_{s=1} \right) \left( 0-\B{i}(1) \right) \\
%
%
%
\left(  \jmp{\phi}, \avg{ D\partial_n \phi^*}\right)_{c+1/2} &= &\left( \widetilde{\phi}_{c+1}\bigg \lvert_{s=-1} - \widetilde{\phi}_{c} \bigg \lvert_{s=1} \right) \frac{1}{2}\left( 0 +  \frac{2 D_c }{\Delta x_c} \frac{\p \B{i}}{\p s} \bigg \lvert_{s=1} \right)  \\
%
%
%
\left( \avg{D\partial_n \phi}, \jmp{\phi^*} \right)_{c+1/2} &=& \frac{1}{2}\left(\frac{2D_{c} }{\Delta x_{c}} \frac{\p \widetilde{\phi}_{c}}{\p s} \bigg \lvert_{s=1} +  \frac{2D_{c+1} }{\Delta x_{c+1}} \frac{\p \widetilde{\phi}_{c+1}}{\p s} \bigg \lvert_{s=-1}  \right) \left( 0- \B{i}(1) \right)
\eeanum

In the above, we have implicitly assumed that the edge normal vector, $\vec{n}$ is oriented in the $+x$ direction.
On the cell interior, this orientation is arbitrary, so long as it is consistent.  
However, on the global boundaries, the edge normal vector must be oriented outward. 
On the left boundary, this is in the $-x$ direction, and on the right boundary, in $+x$ direction.
It is possible to consider the MIP equations in matrix form within each cell.  First, let us define the following $N_P \times N_P$ diagonal matrices:
\beanum
\mathbf{B}_{L,ii} &=& \B{i}(-1) \\
\mathbf{B}_{R,ii} &=& \B{i}(1)  \\
\widehat{\mathbf B}_{L,ii} &=& \frac{\p \B{i}}{\p s} \bigg \lvert_{s=-1}\\
\widehat{\mathbf B}_{R,ii} &=& \frac{\p \B{i}}{\p s} \bigg \lvert_{s= 1} \pep
\eeanum
We also define the following $N_P \times N_P$ matrices:
\benum
\begin{array}{c}
\mathbf{P}_{L,i} \vspace{0.15in}\\
\mathbf{P}_{R,i} \vspace{0.2in}\\
\widehat{\mathbf P}_{L,i} \vspace{0.25in}\\
\widehat{\mathbf P}_{R,i}
\end{array}
\begin{array}{c}
= \vspace{0.15in} \\
= \vspace{0.20in} \\
= \vspace{0.25in} \\
= \\
\end{array}
\begin{array}{cccccc}
\Big [ & B_1(-1) & B_2(-1) & \dots & B_{N_P}(-1) & \Big ] \vspace{0.1in}\\
\Big [ & B_1(1) & B_2(1) & \dots & B_{N_P}(1) & \Big ] \vspace{0.1in}\\
\bigg [ & \frac{\p B_1}{\p s}\bigg \lvert_{s=-1} & \frac{\p B_2}{\p s}\bigg \lvert_{s=-1} & \dots & \frac{\p B_{N_P}}{\p s}\bigg \lvert_{s=-1} & \bigg ] \vspace{0.1in}\\
\bigg [ & \frac{\p B_1}{\p s}\bigg \lvert_{s=1} & \frac{\p B_2}{\p s}\bigg \lvert_{s=1} & \dots & \frac{\p B_{N_P}}{\p s}\bigg \lvert_{s=1} & \bigg ]
\end{array} \pep
\eenum
We note that $\mathbf{P}_L$ and $\mathbf{P}_R$ have only $N_P$ non-zero entries if there are interpolation points located at $s=\pm1$.
The volumetric MIP moment equations, \eqt{eq:vol1} and \eqt{eq:vol2} become:
\benum
\label{eq:vol}
\frac{\Delta x_c}{2} \mathbf{R}_{\sigma_a} \vec{\phi}_c + \frac{2}{\Delta x_c}\mathbf{L}_{D} \vec{\phi}_c \pec 
\eenum
where:
\beanum
\mathbf{R}_{\sigma_a,ij} &=& \int_{-1}^1{\sigma_a(s) \B{i}(s) \B{j}(s) ~ds} \\
\mathbf{L}_{D,ij} &=& \int_{-1}^1{ D(s) \frac{\p \B{i}}{\p s}  \frac{\p \B{j}}{\p s}~ds} \pep
\eeanum
The left interior edge terms in matrix form are:
\beanum
\left( \kappa_{c-1/2} \jmp{\phi},\jmp{\phi^*}\right)_{c-1/2} &=& \kappa_{c-1/2}\mathbf{B}_L \mathbf{P}_L \vec{\phi}_{c} - \kappa_{c-1/2}\mathbf{B}_L \mathbf{P}_R \vec{\phi}_{c-1} \\
\left(  \jmp{\phi}, \avg{ D\partial_n \phi^*}\right)_{c-1/2} &=& \frac{D_c}{\Delta x_c} \bigg \lvert_{s=-1} \widehat{\mathbf B}_L \left[ \mathbf{ P}_L \vec{\phi}_{c} - \mathbf{P}_R \vec{\phi}_{c-1}  \right] \\
\left( \avg{D\partial_n \phi}, \jmp{\phi^*} \right)_{c-1/2} &=& \mathbf{B}_L\left( \frac{D_c}{\Delta x_c}\bigg \lvert_{s=-1} \widehat{\mathbf P}_L\vec{\phi}_c + \frac{D_{c-1}}{\Delta x_{c-1}} \bigg \lvert_{s=1} \widehat{\mathbf P}_R \vec{\phi}_{c-1}   \right)\pep
\eeanum
The right edge terms are:
\beanum
\left( \kappa_{c+1/2} \jmp{\phi},\jmp{\phi^*}\right)_{c+1/2} &=& -\kappa_{c+1/2}\mathbf{ B}_R \mathbf{P}_L \vec{\phi}_{c+1} + \kappa_{c+1/2}\mathbf{ B}_R \mathbf{P}_R \vec{\phi}_c  \\
\left(  \jmp{\phi}, \avg{ D\partial_n \phi^*}\right)_{c+1/2} &=& \frac{D_c}{\Delta x_c }\bigg \lvert_{s=1} \widehat{\mathbf B}_R\left[ \mathbf{P}_L \vec{\phi}_{c+1} - \mathbf{P}_R \vec{\phi}_c  \right] \\
\left( \avg{D\partial_n \phi}, \jmp{\phi^*} \right)_{c+1/2} &=& -\mathbf{ B}_R\left[ \frac{D_c}{\Delta x_c} \bigg \lvert_{s=1} \widehat{\mathbf P}_R \vec{\phi}_c+ \frac{D_{c+1}}{\Delta x_{c+1}} \bigg \lvert_{s=-1} \widehat{\mathbf P}_L \vec{\phi}_{c+1} \right]\pep
\eeanum
%
%\begin{multline}
%\kappa_{c-1/2} \mathbf{B}_L \left(\mathbf{P}_L \vec{\phi}_c - \mathbf{P}_R \vec{\phi}_{c-1}  \right)
%+   \frac{D_c}{\Delta x_c} \bigg \lvert_{s=-1} \widehat{\mathbf B}_L \left( \mathbf{P}_L\vec{\phi}_c - \mathbf{P}_R \vec{\phi}_{c-1} \right) \\
%+ \mathbf{B}_L \left( \frac{D_{c-1}}{\Delta x_{c-1}} \bigg \lvert_{s=1} \widehat{\mathbf P}_R \vec{\phi}_{c-1} + \frac{D_c}{\Delta x_c} \bigg \lvert_{s=-1}  \widehat{\mathbf P}_L \vec{\phi}_c  \right) \pep
%\label{eq:left_e}
%\end{multline}
%
The interior cell MIP equations are:
\begin{multline*}
\left( \sigma_a\phi,\phi^* \right)_c + \left( D\del\phi,\del\phi \right)_c
 +\left( \kappa_{c-1/2} \jmp{\phi},\jmp{\phi^*}\right)_{c-1/2} 
 +\left(  \jmp{\phi}, \avg{ D\partial_n \phi^*}\right)_{c-1/2} 
 +\left( \avg{D\partial_n\phi}, \jmp{\phi^*} \right)_{c-1/2} \\
  +\left( \kappa_{c+1/2} \jmp{\phi},\jmp{\phi^*}\right)_{c+1/2} 
 +\left(  \jmp{\phi}, \avg{ D\partial_n \phi^*}\right)_{c+1/2} 
 +\left( \avg{D\partial_n\phi}, \jmp{\phi^*} \right)_{c+1/2} \pep
\end{multline*}
%
%
Combining, the interior cell MIP equations written in matrix form are:
\begin{multline}
\left[ -\kappa_{c-1/2}\mathbf{B}_L\mathbf{P}_R - \frac{D_c}{\Delta x_c} \bigg \lvert_{s=-1} \widehat{\mathbf B}_L  \mathbf{P}_R + \mathbf{B}_L \frac{D_{c-1}}{\Delta x_{c-1}}\bigg \lvert_{s=1} \widehat{\mathbf P}_R \right]\vec{\phi}_{c-1} \\
%
%
\left[\frac{\Delta x_c}{2}\mathbf{R}_{\sigma_a} + \frac{2}{\Delta x_c}\mathbf{L}_D + \kappa_{c-1/2}\mathbf{B}_L\mathbf{P}_L + \frac{D_c}{\Delta x_c} \bigg \lvert_{s=-1} \widehat{\mathbf B}_L \mathbf{P}_L  + \mathbf{B}_L \frac{D_c}{\Delta x_c} \bigg \lvert_{s=-1} \widehat{\mathbf P}_L \dots \right. \\
\left. + \kappa_{c+1/2}\mathbf{B}_R \mathbf{P}_R - \frac{D_c}{\Delta x_c}\bigg \lvert_{s=1}\widehat{\mathbf B}_R \mathbf{P}_R -\mathbf{ B}_R \frac{D_c}{\Delta x_c}\bigg \lvert_{s=1} \widehat{\mathbf P}_R \right] \vec{\phi}_c \\
%
%
\left[ -\kappa_{c+1/2}\mathbf{B}_R \mathbf{P}_L + \frac{D_c}{\Delta x_c}\bigg \lvert_{s=1}\widehat{\mathbf B}_R \mathbf{P}_L - \mathbf{ B}_R \frac{D_{c+1}}{\Delta x_{c+1}} \bigg \lvert_{s=-1} \widehat{\mathbf P}_L \right] \vec{\phi}_{c+1}
\end{multline}
%
%%%%The right edge terms are:
%%%%\begin{multline}
%%%%-\kappa_{c+1/2}\mathbf{B}_R\left( \mathbf{P}_L \vec{\phi}_{c+1} - \mathbf{P}_R \vec{\phi}_{c} \right) +
%%%%\frac{D_c}{\Delta x_c} \bigg\lvert_{s=1} \widehat{\mathbf B}_R \left(\mathbf{P}_L \vec{\phi}_{c+1} - \mathbf{P}_R \vec{\phi}_c  \right) \\
%%%%-\mathbf{B}_R \left(\frac{D_c}{ \Delta x_c} \bigg \lvert_{s=1} \widehat{\mathbf P}_R \vec{\phi}_c + \frac{D_{c+1} }{\Delta x_{c+1}} \bigg \lvert_{s=-1} \widehat{\mathbf P}_L \vec{\phi}_{c+1}  \right) \pep
%%%%\label{eq:right_e}
%%%%\end{multline}
%%%%%
%%%%%
%%%%%
%%%%Combining \eqt{eq:vol}, \eqt{eq:left_e}, and \eqt{eq:right_e} and simplifying we have:
%%%%%
%%%%%
%%%%\begin{multline}
%%%%-\left[ \kappa_{c-1/2} \mathbf{B}_L \mathbf{P}_R + \frac{ D_c}{\Delta x_c}\big \lvert_{s=-1}\widehat{\mathbf B}_L \mathbf{P}_R
%%%%-\frac{D_{c-1}}{\Delta x_{c-1}}\bigg \lvert_{s=1} \mathbf{B}_L \widehat{\mathbf P}_R \right]\vec{\phi}_{c-1} \\
%%%%%
%%%%%
%%%%+\left[\frac{\Delta x_c}{2} \mathbf{R}_{\sigma_a} + \frac{2}{\Delta x_c}\mathbf{L}_D + \kappa_{c-1/2}\mathbf{B}_L \mathbf{P}_L 
%%%%+ \frac{D_c}{\Delta x_c}\bigg \lvert_{s=-1}\left( \widehat{\mathbf B}_L \mathbf{P}_L + \mathbf{B}_L \widehat{\mathbf P}_L \right)
%%%% \dots \right. \\ \left. 
%%%% + \kappa_{c+1/2} \mathbf{B}_R \mathbf{P}_R - \frac{D_c}{\Delta x_c}\bigg \lvert_{s=1}\left(  \widehat{\mathbf B}_R\mathbf{P}_R + \mathbf{B}_R \widehat{\mathbf P}_R \right)\right]\vec{\phi}_c \\
%%%%%
%%%%%
%%%%-\left[\kappa_{c+1/2}\mathbf{B}_R \mathbf{P}_L - \frac{D_c}{\Delta x_c} \bigg\lvert_{s=1}\widehat{\mathbf B}_R \mathbf{P}_L +
%%%%\frac{D_{c+1}}{\Delta x_{c+1}} \bigg \lvert_{s=-1}\mathbf{B}_R \widehat{\mathbf P}_L \right]\vec{\phi}_{c+1} \pep
%%%%\end{multline}

\subsection{Boundary Terms}

We now consider the shared boundary terms:
\be
(\kappa_{e}\phi,\phi^*)_{\partial \mathcal D} -\frac{1}{2}(\phi ,D\partial_n \phi^*)_{\partial \mathcal D}
\ee

\subsubsection{Leftmost Cell}

\beanum
\left( \kappa_{1/2} \phi, \phi^*\right)_{1/2} &=& \kappa_{1/2}\left(\widetilde{\phi}_1(-1) \right) \B{i}(-1) \\
-\frac{1}{2}\left( \phi, D\partial_n \phi^* \right)_{1/2} &=& -  \frac{1}{2}\widetilde{\phi}_1(-1) \left(-\frac{2 D_1}{\Delta x_1} \frac{\p \B{i}}{\p s} \bigg \lvert_{s=-1} \right)\\
\kappa_{1/2} &=& \max \left( \frac{1}{4},\kappa_{1/2}^{IP} \right) \\
\kappa_{1/2}^{IP} &=& 2 p_1 (p_1 + 1)\frac{D_1}{\Delta x_1}\bigg \lvert_{s=-1} \pep
\eeanum
The (shared component of the) leftmost cell moment equations are:
\begin{multline*}
\left( \sigma_a\phi,\phi^* \right)_1 + \left( D\del\phi,\del\phi \right)_1
  +\left( \kappa_{3/2} \jmp{\phi},\jmp{\phi^*}\right)_{3/2} 
 +\left(  \jmp{\phi}, \avg{ D\partial_n \phi^*}\right)_{3/2} \\
 +\left( \avg{D\partial_n\phi}, \jmp{\phi^*} \right)_{3/2} 
 + (\kappa_{1/2} \phi, \phi^*)_{1/2} - \frac{1}{2}\left( \phi,D\partial_n \phi^*  \right)_{1/2}   \pep
\end{multline*}
%
%
Written in matrix form:
\begin{multline}
\left[ \frac{\Delta x_1}{2} \mathbf{R}_{\sigma_a} + \frac{2}{\Delta x_1}\mathbf{L}_D + \kappa_{3/2} \mathbf{B}_R \mathbf{P}_R - \frac{D_1}{\Delta x_1}\bigg \lvert_{s=1}\widehat{\mathbf B}_R \mathbf{P}_R - \frac{D_1}{\Delta x_1}\bigg \vert_{s=1} \mathbf{B}_R \widehat{\mathbf P}_R \right. \dots \\
\left. + \kappa_{1/2} \mathbf{B}_L \mathbf{P}_L + \frac{D_1}{\Delta x_1} \bigg \lvert_{s=-1} \widehat{\mathbf B}_L \mathbf{P}_L \right]\vec{\phi}_1 \\
%
+ \left[ -\kappa_{3/2}\mathbf{B}_R\mathbf{P}_L + \frac{D_1}{\Delta x_1}\bigg \lvert_{s=1}\widehat{\mathbf B}_R \mathbf{P}_L -\frac{D_2}{\Delta x_2}\bigg \vert_{s=-1} \mathbf{B}_R \widehat{\mathbf P}_L\right] \vec{\phi}_2
\end{multline}

\subsubsection{Rightmost Cell}
%
The boundary terms are:
\beanum
\left( \kappa_{N+1/2} \phi, \phi^*\right)_{N+1/2} &=& \kappa_{N+1/2}\left(\widetilde{\phi}_N(1) \right) \B{i}(1) \\
%
%
-\frac{1}{2}\left( \phi, D\partial_n \phi^* \right)_{N+1/2} &=&  -\frac{1}{2}\widetilde{\phi}_N(1) \left(\frac{2 D_N}{\Delta x_N} \frac{\p \B{i}}{\p s} \bigg \lvert_{s=1} \right)\\
%
%
%
\kappa_{N+1/2} &=& \max \left( \frac{1}{4},\kappa_{N+1/2}^{IP} \right) \\
%
%
\kappa_{N + 1/2}^{IP} &=& 2 p_N \frac{D_N(1)}{\Delta x_N} \pep
\eeanum
The shared components of the rightmost cell moment equations are:
\begin{multline*}
\left( \sigma_a\phi,\phi^* \right)_N + \left( D\del\phi,\del\phi \right)_N
 +\left( \kappa_{N-1/2} \jmp{\phi},\jmp{\phi^*}\right)_{N-1/2} 
 +\left(  \jmp{\phi}, \avg{ D\partial_n \phi^*}\right)_{N-1/2} 
 +\left( \avg{D\partial_n\phi}, \jmp{\phi^*} \right)_{N-1/2} \\
  +\left( \kappa_{N+1/2} \phi,\phi^*\right)_{N+1/2}
 -\frac{1}{2}\left( \phi,  D\partial_n \phi^*\right)_{N+1/2} \pep
\end{multline*}
Written in matrix form:
\begin{multline}
\left[ -\kappa_{N-1/2}\mathbf{B}_L \mathbf{P}_R -\frac{D_N}{\Delta x_N} \bigg \lvert_{s=-1} \widehat{\mathbf B}_L \mathbf{P}_R + \frac{D_{N-1}}{\Delta x_{N-1}}\bigg \lvert_{s=1} \mathbf{B}_L \widehat{\mathbf P}_R \right] \vec{\phi}_{N-1} \\
\left[ \frac{\Delta x_N}{2} \mathbf{R}_{\sigma_a} + \frac{2}{\Delta x_N} \mathbf{L}_D + \kappa_{N-1/2} \mathbf{B}_L \mathbf{P}_L + \frac{D_N}{\Delta x_N}\bigg\lvert_{s=-1} \widehat{\mathbf B}_L \mathbf{P}_L + \frac{D_N}{\Delta x_N} \bigg \lvert_{s=-1} \mathbf{B}_L \widehat{\mathbf P}_L  \dots \right. \\
\left.
+ \kappa_{N+1/2} \mathbf{B}_R \mathbf{P}_R - \frac{D_N}{\Delta x_N} \bigg \lvert_{s=1} \widehat{\mathbf B}_R \mathbf{P}_R
\right] \vec{\phi}_N
\end{multline}

\section{Different Boundary Terms}

\subsection{M\&C Paper}
If the last boundary term is:
\benum
-\frac{1}{2}\left(D \partial_n \phi,\phi^*  \right)_{\partial \mathcal D}
\eenum
all boundary terms affect only the boundary cell equations.
We now define the terms that are added to the left and rightmost cell moment equations.

\subsubsection{Leftmost Cell}
\benum
-\frac{1}{2}\left(D \partial_n \phi,\phi^*  \right)_{\partial \mathcal D} = -\frac{1}{2} \frac{-2D_1}{\Delta x_1}\bigg\lvert_{s=-1}  \frac{d \widetilde{\phi}_1}{d s} B_i(-1) \pec
\eenum
which in matrix form is:
\benum
\frac{D_1}{\Delta x_1}\bigg \lvert_{s=-1} \mathbf{B}_L \widehat{\mathbf P}_L \vec{\phi}_1
\eenum

\subsubsection{Rightmost Cell}
\benum
-\frac{1}{2}\left(D \partial_n \phi,\phi^*  \right)_{\partial \mathcal D} = -\frac{1}{2} \frac{2D_N}{\Delta x_N}\bigg\lvert_{s=1}  \frac{d \widetilde{\phi}_N}{d s} B_i(1) \pec
\eenum
which in matrix form is:
\benum
-\frac{D_N}{\Delta x_N}\bigg \lvert_{s=1} \mathbf{B}_R \widehat{\mathbf P}_R \vec{\phi}_N
\eenum

\subsection{NS\&E Paper}
If the last boundary term is:
\benum
-\frac{1}{2}\left(D\partial_n\phi ,\phi \right)_{\partial \mathcal D}
\eenum
we are in big trouble as this looks non-linear.

\subsubsection{Leftmost Cell}
\benum
-\frac{1}{2}\left(D \partial_n \phi,\phi  \right)_{\partial \mathcal D} = -\frac{1}{2} \frac{-2D_1}{\Delta x_1}\bigg\lvert_{s=-1}  \frac{d \widetilde{\phi}_1}{d s} \widetilde{\phi}_1(-1) \pec
\eenum

\subsubsection{Rightmost Cell}
\benum
-\frac{1}{2}\left(D \partial_n \phi,\phi  \right)_{\partial \mathcal D} = -\frac{1}{2} \frac{2D_N}{\Delta x_N}\bigg\lvert_{s=1}  \frac{d \widetilde{\phi}_N}{d s} \widetilde{\phi}_N(1) \pec
\eenum

\begin{thebibliography}{20}

\bibitem{M4S} M. L. Adams and W. R. Martin , ``Diffusion Synthetic Acceleration of Discontinuous Finite Element Transport Iterations,'' {\it Nuclear Science and Engineering}, {\bf 111}, pp. 145-167 (1992).

\end{thebibliography}

\end{document}