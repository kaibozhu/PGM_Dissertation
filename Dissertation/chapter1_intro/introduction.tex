%%%%%%%%%%%%%%%%%%%%%%%%%%%%%%%%%%%%%%%%%%%%%%%%%%%
%
%  New template code for TAMU Theses and Dissertations starting Fall 2012.  
%  For more info about this template or the 
%  TAMU LaTeX User's Group, see http://www.howdy.me/.
%
%  Author: Wendy Lynn Turner 
%	 Version 1.0 
%  Last updated 8/5/2012
%
%%%%%%%%%%%%%%%%%%%%%%%%%%%%%%%%%%%%%%%%%%%%%%%%%%%

%%%%%%%%%%%%%%%%%%%%%%%%%%%%%%%%%%%%%%%%%%%%%%%%%%%%%%%%%%%%%%%%%%%%%%
%%                           SECTION I
%%%%%%%%%%%%%%%%%%%%%%%%%%%%%%%%%%%%%%%%%%%%%%%%%%%%%%%%%%%%%%%%%%%%%


\pagestyle{plain} % No headers, just page numbers
\pagenumbering{arabic} % Arabic numerals
\setcounter{page}{1}


\chapter{\uppercase {Introduction}}
\label{sec:introduction}
This dissertation is dedicated to the solution of the thermal radiative transfer (TRT) equations.  The TRT equations:
\begin{subequations}
\label{eq:full_trt}
\beanum
\frac{1}{c} \frac{\p I}{\p t} + \omg \cdot \vec{\nabla} I  + \sigma_t I &=& \int_0^{\infty}{ \int_{4\pi}{ \sigma_s(\omg'\to\omg,E'\to E) I , d\omg' } dE'}+ \sigma_a B \\
C_v \frac{\p T}{\p t} &=& \int_0^{\infty}{  \sigma_a \left( \phi- 4\pi B   \right) dE} \pec 
\eeanum
\end{subequations}
are a nonlinear system of equations that describe the exchange of energy between a photon radiation field and a non-moving material.  
Radiation intensity, $I$, is a seven dimensional field dependent upon spatial location, $\vec{x}$; photon energy, $E$; photon direction of travel, $\omg$; and time $t$.  
$c$ is the speed of light.
Material opacities for all interactions, $\sigma_t$; absorption, $\sigma_a$; and  scattering, $\sigma_s$ are functions of photon energy and material temperature, $T$. 
Material heat capacity, $C_v$, is also a function of material temperature.  
The angle integrated radiation intensity, $\phi$, is an integral over all photon directions of the the photon intensity and is a function of space and photon energy.  Finally, the Planck function, $B$, is a function of photon energy and material temperature.
While materials at all temperatures emit photon radiation, the radiation emission is proportional to $T^4$.  
Thus, solution of the radiative transfer equations is most important in situations where materials are very hot.  
Solving the thermal radiative transfer equations is an important component of the simulation of different scientific and engineering problems including astrophysics supernova explosions and high energy density laboratory physics experiments like those conducted at the National Ignition Facility.

\section{Simplifications of the Thermal Radiative Transfer Equations}
In this dissertation, we we make a number of simplifying assumptions to make solution of \eqts{eq:full_trt} more tractable.  
First, we limit our focus to 1-D Cartesian (slab) geometry.
The assumption of slab geometry is not required, but slab geometry radiation transport simulations require significantly less computational time.
Further, any methods that have a possibility of being viable for radiation transport in multiple spatial dimensions must also work well in slab geometry.

Second, we approximate the continuous angle dependence of the intensity using the discrete ordinates ($S_N$) method.
The $S_N$ method approximates the true definition of the angle integrated intensity,
\be
\phi(\vec{x},E,t) = \int_{4\pi}{I(\vec{x},\omg,E,t) d\omg} \pec
\ee
using quadrature integration,
\benum
\phi(\vec{x},E,t) \approx \sum_{d=1}^{N_{dir}}{ w_d I(\vec{x},\omg_d,E,t) } \pep
\label{eq:sn_def}
\eenum
In \eqt{eq:sn_def}, $\{w_d,\omg_d \}_{d=1,\dots N_{dir}}$ is the set of $N_{dir}$ quadrature weights $w_d$ and discrete directions, $\omg_d$ and corresponding intensities $I_d$.

Finally, we  treat the photon energy dependence assuming a grey, or photon energy integrated model.  
The grey assumption assumes suitable opacities, $\sigma_{grey}$ exist such that
\begin{subequations}
\label{eq:chap1_opacity}
\beanum
\int_0^{\infty}{ \sigma(E) I(E) dE} &=& \sigma_{grey} I_{grey} \\
 \int_0^{\infty}{ \sigma(E) \phi(E) dE} &=& \sigma_{grey} \phi_{grey} \\
\int_0^{\infty}{ \sigma(E) B(E,T) dE} &=& \sigma_{grey} B_{grey}(T) \pec
\eeanum
\end{subequations}
where $I_{grey}$, $\phi_{grey}$, and $B_{grey}$ are photon energy integrated quantities,
\beanum
I_{grey} &=& \int_0^{\infty}{ I(E) dE}  \\
\phi_{grey} &=& \int_0^{\infty}{ \phi(E) dE} \\
B_{grey} &=& \int_0^{\infty}{ B(E,T) dE} \pep
\eeanum
For the remainder of this work, we will forgo explicitly denoting quantities as grey, and unless otherwise noted, all quantities should be assumed to be photon energy integrated (grey).
Though the assumption of \eqts{eq:chap1_opacity} does not hold unless all opacities are constant in energy, methods developed for the grey case are readily extensible
to the multi-frequency treatment of photon energy dependence, and the multi-frequency approximation is by far the most common treatment of thermal radiative transfer photon energy dependence\cite{lewis_book}.

\section{Spatial and Temporal Discretization}

To complete a description of the approach we will take to solve \eqts{eq:full_trt}, we now describe how we will discretize the spatial and temporal variables.

\subsection{Time Integration}

The appearance of the speed of light in \eqt{eq:full_trt} results in the TRT equations being very stiff.
To solve the such a stiff system of equations would require either an impractically small time step, or the use of implicit methods.
We elect to use Diagonally Implicit Runge-Kutta (DIRK) methods to advance our TRT solution in time.
The simplest of DIRK scheme is the first order implicit Euler scheme, but more advanced DIRK higher order methods in time \cite{alexander}.

\subsection{Spatial Discretization with Discontinuous Finite Elements}

The linear discontinuous finite element method (LDFEM) has long been used to solve the discrete ordinates neutron transport equation \cite{reed}.
Through manipulation, the thermal radiative transfer equations can be transformed  into a form that is equivalent to the neutron transport equation with pseudo- scattering, fission, and fixed sources.
This makes it possible to use the same methods and techniques developed for neutron transport to assist in solving the  thermal radiative transfer equations.
LDFEM has achieved wide spread acceptance in the neutron transport community because it is accurate \cite{larsen_nelson} and highly damped.  
Because it possesses the thick diffusion limit \cite{larsen_morel_asymptotics}, LDFEM has also been applied to the $S_N$ TRT equations.  
Morel, Wareing, and Smith first considered the application of LDFEM to the $S_N$ TRT equations in \cite{morel_radtran}.
Mass matrix lumped LDFEM was shown to preserve the thick equilibrium diffusion limit \cite{morel_radtran}.  
This suggests that discontinuous finite element (DFEM) schemes can be used to accurately solve the TRT equations in both diffusive and transport effects dominated regions.

\section{Progression Towards Higher Order DFEM Thermal Radiative Transfer}

For higher order (quadratic and higher polynomial degree trial spaces) DFEM to be accurate and practical for solving \eqts{eq:full_trt} we must demonstrate that higher order DFEM:
\begin{enumerate}
\item are ``robust'',
\item account for within cell spatial variation of opacity accurately, and
\item can be accelerated using appropriate iterative acceleration techniques.
\end{enumerate}
By ``robust'', we mean that that calculated radiation outflow from a spatial cell is strictly positive for all cell widths and optical thicknesses.

In \chapref{sec:chapter2_constant_xs} we use a steady-state, mono-energetic, source-free pure absorber neutron transport problem with a cross section that is constant in space to examine the robustness of different radiation transport DFEM matrix lumping techniques.
Next, we extend the techniques developed by Adams \cite{adams_scb,adams_nowak}, for a spatial discretization scheme related to LDFEM  to address the within cell spatial variation of opacity, for higher order DFEM in \chapref{sec:chapter3_variable_xs} .
%We study the accuracy and robustness of a technique that explicitly accounts for the spatial variation of neutron transport interaction cross sections within each spatial cell
%for arbitrary trial space degree  DFEM in 
Then, we examine iterative acceleration techniques compatible with higher order DFEM spatial discretizations in Chapter \ref{sec:chapter4_acceleration}.
In preparation for solving the coupled, non-linear TRT equations, in \chapref{sec:chapter5_depletion} we combine all of the strategies we have developed in Chapters \ref{sec:chapter2_constant_xs}-\ref{sec:chapter4_acceleration} and apply them to a coupled system of linear equations to solve a two-group fuel depletion problem that uses explicit Euler time differencing.
Finally, in \secref{sec:chapter6_grey_radtran} we solve the grey thermal radiative transfer equations using higher order DFEM, fully deriving the necessary  equations, acceleration techniques, and providing numerical results that verify our newly developed DFEM methods and their implementation. 

